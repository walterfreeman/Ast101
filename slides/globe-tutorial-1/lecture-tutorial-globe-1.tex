\documentclass[12pt]{article}
\setlength\parindent{0pt}
\usepackage{amsmath}
\usepackage{graphicx}
\usepackage{fullpage}
\setlength{\parskip}{4mm}
\def\LL{\left\langle}   % left angle bracket
\def\RR{\right\rangle}  % right angle bracket
\def\LP{\left(}         % left parenthesis
\def\RP{\right)}        % right parenthesis
\def\LB{\left\{}        % left curly bracket
\def\RB{\right\}}       % right curly bracket
\def\PAR#1#2{ {{\partial #1}\over{\partial #2}} }
\def\PARTWO#1#2{ {{\partial^2 #1}\over{\partial #2}^2} }
\def\PARTWOMIX#1#2#3{ {{\partial^2 #1}\over{\partial #2 \partial #3}} }
\newcommand{\BE}{\begin{displaymath}}
\newcommand{\EE}{\end{displaymath}}
\newcommand{\BNE}{\begin{equation}}
\newcommand{\ENE}{\end{equation}}
\newcommand{\BEA}{\begin{eqnarray}}
\newcommand{\EEA}{\nonumber\end{eqnarray}}
\newcommand{\EL}{\nonumber\\}
\newcommand{\la}[1]{\label{#1}}
\newcommand{\ie}{{\em i.e.\ }}
\newcommand{\eg}{{\em e.\,g.\ }}
\newcommand{\cf}{cf.\ }
\newcommand{\etc}{etc.\ }
\newcommand{\Tr}{{\rm tr}}
\newcommand{\etal}{{\it et al.}}
\newcommand{\OL}[1]{\overline{#1}\ } % overline
\newcommand{\OLL}[1]{\overline{\overline{#1}}\ } % double overline
\newcommand{\OON}{\frac{1}{N}} % "one over N"
\newcommand{\OOX}[1]{\frac{1}{#1}} % "one over X"
\pagenumbering{gobble}
\begin{document}
\Large
\centerline{\sc{Lecture Tutorial -- Inflatable Earths}}

\normalsize

{\it Note: You should do all these activities for yourself, with your own hands, with your own globe. However, you should 
talk to your friends around you to help you figure out the answers to these questions. 

Remember, you may bring 
these globes to the first exam, write things on them, and manipulate them however you need to help you solve problems.}


\section{Getting Started}
\begin{enumerate}

\item Inflate your blow-up Earths. We don't have any flat-Earthers in here, do we? :)

\item Find Syracuse University on the globe. Stick one of your little googly-eyes on it. Nobody ever said we were serious in 
here!

\item Turn the globe so that it is oriented just like the real Earth is right now. Do this in two steps:

\begin{enumerate}

\item Figure out what location on your globe ought to be pointed straight upward. What point is that? Turn your globe
so that this location is pointed upward.

\vspace{1in}

\item {Using a Sharpie or pen, draw arrows starting at that point and pointing north, south, east, and west. (Remember: north, 
south, east, and west are directions on the surface of the Earth. Directions in the sky combine one these directions with
an elevation above the horizon, so something like ``in the northern sky, close to the horizon'') 

\medskip

Then, keeping that point straight up, rotate the Earth so that the arrow labeled ``north'' on your globe faces 
the actual direction north (toward Hendricks Chapel). Make sure you agree with your group how the Earth should be oriented
before you move on.
}

\end{enumerate}
\end{enumerate}

\section{Describing the Sky -- Now}

In this section, you are going to imagine that you are an observer in Syracuse on the globe, 
and that the objects in the classroom are stars in space. 

\begin{enumerate}
\item The North Celestial Pole -- where the star Polaris, called the North Star, is -- is directly above the North Pole.
Point to it in the room. Then, imagine that you are an observer in Syracuse on your globe. Where would the North Star appear
in your sky?

\vspace{1in}

\item Some objects in space will be below your observer's horizon. A simple way to imagine the horizon is as follows: take a 
flat piece of paper or a thin book, and lay it flat on your globe at the observer's location. Imagine that this
object extends along the same plane out in all directions. Since your ``observer'' is so incredibly small
compared to the size of the globe, anything {\it above} the horizon is visible, and anything {\it below} the horizon 
won't be visible since the line of sight to it would have to pass through the Earth. 

\item Imagine now that there is a star located in the door that is on your right and behind you. (We'll call this star
``$\alpha$~Portis'' -- the first Door Star.) Imagine yourself again as a little ant in Syracuse on your globe, looking up
at the heavens; where is $\alpha$~Portis in your sky? Remember that a description of the position of a star requires two
things: a direction (North, Northeast, etc.) and an elevation (``close to the horizon'', ``nearly overhead'' , ``high in the sky'',
etc.) 

\vspace{1in}

\item Now, imagine a second star $\beta$~Portis, located in the door in the front of the classroom to the left, by the computer. 
Where is that star in your sky?

\vspace{1in}

\end{enumerate}

\section{The Rotation of the Earth, I}

\begin{enumerate}
\item The Earth rotates along an axis running from the South Pole to the North Pole. You'll need to figure out which
direction it rotates. Remember that, as the Earth rotates, objects appear above the eastern horizon and disappear above the
western horizon.

Two students, Merry and Pippin, are arguing about which way the Earth should rotate.

{\bf Merry:} I think the Earth has to rotate clockwise when seen from above the North Pole (from East to West). Things rise in 
the East and set in the West, right? So if I look down at Syracuse and spin the Earth from East to West, I see new things 
appearing around the side of the globe in the East (Europe and Africa, say), and old things vanishing in the West, like Hawai'i.

{\bf Pippin:} I don't think that's quite right. We're not concerned with places on Earth appearing or disappearing; we care about 
how an observer on Earth would see the sky. Imagine that some star lies just below the Eastern horizon, and is about to 
rise. In order to spin the Earth so that that star appears over the horizon to the observer, I need to rotate the globe {\it 
toward} the star -- West to East, counter-clockwise when seen from above -- not {\it away} from it like Merry says.

Which one of our pint-sized astronomers is correct? Explain your reasoning.

\vspace{1.5in}

\item As you did before, orient your Earth as it actually is now. 
Imagine that you are an observer in Syracuse (which you've marked with a googly-eye, right?) In a previous part, you 
determined where in your sky the North Star, Polaris, would appear. Now, imagine that six hours pass. Rotate the Earth to
where it will be six hours from now.

Now, according to an observer in Syracuse, where will Polaris appear to be in their sky?

\vspace{1in}

\item Allow another six hours to pass. Where will Polaris be now in the sky of an observer in Syracuse?

\vspace{1in}

\item Why does Polaris appear to move in the sky in the way that it did here?

\vspace{1in}

\item While you're thinking about the North Star, where will it appear in the sky for an observer in Alert, Nunavut (located in
far northern Canada, very near to the North Pole)? Will it be high in the sky? Low in the sky?

\vspace{1in}

\item Consider an observer now in Quito, Ecuador, located right on the equator. Where will they see Polaris? (Add another 
of your googly-eyes to Quito once you find it, on the northwestern part of South America.)

\vspace{1in}

\end{enumerate}

\section{The Rotation of the Earth, II}

 Now, put the Earth back like it was (with Syracuse on top), and think of your star $\alpha$ Portis again. Repeat the same
exercise with that star:

\begin{enumerate}

\item How would you describe the position of $\alpha$ Portis in the sky to an observer at the position of Syracuse? (You already did this)

\vspace{1in}

\item Rotate Earth as though six hours have passed. How would you describe the position of $\alpha$ Portis now? Remember, you 
are considering where the star would appear {\it from the perspective of an observer on your globe} (at the little googly-eye).
\vspace{1in}
\item Rotate Earth as though six more hours (twelve total) have passed. How would you describe the position of $\alpha$ Portis now?
\vspace{1in}
\item Rotate Earth as though six more hours (eighteen total)  have passed. How would you describe the position of $\alpha$ Portis now?
\vspace{1in}

\end{enumerate}

\section{The Earth and the Celestial Sphere}

In the previous section, you said ``Okay, the star doesn't move but the Earth turns'', and based on that 
figured out how a star would move as seen by an observer in Syracuse. This is thinking according to the {\bf modern perspective}
-- one where we are aware that the stars do not move much, and the Earth rotates from the West to the East once per day.

Last week we modeled the night sky as a celestial sphere that rotates counter-clockwise around the North Celestial Pole once per day, carrying
the stars with it. Is this consistent with your answers to the previous part? On the rest of this page, make a diagram the path of
$\alpha$ Portis during one day.





\end{document}

