\documentclass[11pt]{article}
\usepackage{tocloft}
\usepackage{graphicx}
\usepackage{calc}
\usepackage{amssymb}
\usepackage{color}
\usepackage{array}
\usepackage[sc]{mathpazo}
\usepackage{url}
\usepackage[final]{pdfpages}
\usepackage{amsmath}

%\linespread{1.05}
\oddsidemargin=0pt
\evensidemargin=0pt
\textwidth=6.5in
\topmargin=0pt
\headheight=0pt
\headsep=0pt
\textheight=9in
% EXPERIMENTAL
%\parindent=0pt
%\parskip=3pt
\setlength{\parindent}{0cm}
\newcommand\secfont{\fontfamily{cmss}\selectfont}%\textwidth 5.5truein
\newcommand\pifheading[1]{{\secfont\textbf{#1}:}}
%\oddsidemargin -0.40truein
%\textheight 8.0truein
%\topmargin -0.25truein
\def\lo{
\mathrel{\raise.3ex\hbox{$<$}\mkern-14mu\lower0.6ex\hbox{$\sim$}}
}
\def\hi{
\mathrel{\raise.3ex\hbox{$>$}\mkern-14mu\lower0.6ex\hbox{$\sim$}}
}

\textwidth = 6.6 in
\textheight = 9.1 in
\oddsidemargin = -0.05 in
\evensidemargin = +0.05 in
\topmargin = -.1 in
\headheight = 0.0 in
\headsep = 0.0 in
\parskip = 0.06in
\newcommand\registered{{\ooalign{\hfil\raise .00ex\hbox{\scriptsize R}\hfil\crcr\mathhexbox20D}}}

%% Define a new 'leo' style for the package that will use a smaller font.
\makeatletter
\def\url@leostyle{%
  \@ifundefined{selectfont}{\def\UrlFont{\sf}}{\def\UrlFont{\small\ttfamily}}}
\makeatother
%% Now actually use the newly defined style.
\urlstyle{leostyle}

%\pagestyle{empty}
%\includeonly{previous,proposal_references}
%\includeonly{proposal_references}
%\includeonly{previous}

% TOC

\pagenumbering{gobble}

\begin{document}
%%%%%%%%%%%%%%%%%%%%%%%%%%%%%%%%%%%%%%%%%%%%%%%%%%%%%%%%%%%%%%%%%%%%%
\begin{center}
\textbf{\Large
AST101: Our Corner of the Universe \\
\vspace*{0.1cm}
Lab 5: Simulating Planets' Orbits (II): The Three-Body Problem
}
\end{center}

\vspace*{0.5cm}

{\Large Name:}\vspace*{0.5cm}\\\hrule
{\Large Student number (SUID):}\vspace*{0.5cm}\\\hrule
{\Large Lab section:}\vspace*{0.5cm}\\\hrule
{\Large Group Members:}\vspace*{0.5cm}\\\hrule
\vspace*{0.5cm}

%%%%%%%%%%%%%%%%%%%%%%%%%%%%%%%%%%%%%%%%%%%%%%%%%%%%%%%%%%%%%%%%%%%%%

\section{Objective}

To use a computer simulation to explore the properties of the gravitational motion of three objects at once.

\section{Introduction}

Isaac Newton was able to use the laws of motion to solve -- with pen and paper -- for the *exact* motions of two objects orbiting each other. But it is possible to prove that there is {\it no} pencil and paper solution (using algebra and calculus with variables )for three objects. We have to resort to calculations with numbers using a computer.

This situation is called the {\it three-body problem}.

\section{New Simulators}

I've modified the computer simulation we used previously so that it can handle the motion of three objects at once. This requires us to think carefully about Newton's law of motion a little bit, since it is the {\it combined} force from all of the objects that gives the acceleration.

Suppose we are thinking about the motion of the Earth, the Sun, and the planet Jupiter together. In symbols, we can say

$$
\mbox{Acceleration of Earth} = \frac{\mbox{Force on Earth from the Sun's gravity + Force on Earth from Jupiter's gravity}}{\mbox{Mass of Earth}}
$$

We have several different simulators this time. One of them is intended for two planets around one star, one of them is intended for one planet around two stars, and one of them is intended for three stars, but they all work the same: the only difference is what the variables are called and which things the simulator graphs for you.





\newpage

\section{The Earth, the Sun, and Other Planets}

Here is some background information on the planets:


	\begin{tabular}{|l|l|l|l|}
		\hline
		& Earth     & Venus     & Jupiter \\ \hline
		Aphelion (AU)       & 1.017     & 0.728     & 5.46    \\ \hline
		Perihelion (AU)     & 0.983     & 0.718     & 4.95    \\ \hline
		Eccentricity        & 0.017     & 0.007     & 0.049   \\ \hline
		Period (years)      & 1         & 0.62      & 11.86   \\ \hline
		Mass (solar masses) & 0.0000030 & 0.0000024 (0.8 Earths) & 0.00096 (318 Earths)\\ \hline
	\end{tabular}



For your prelabs this week and last, you simulated the orbits of Earth, Venus, and Jupiter around the Sun, each planet by itself.

Newton showed that, if {\it the only objects that exist are the Sun and one planet}, that the planet will orbit the Sun in the same orbit, with the same eccentricity and period, forever. However, we now need to think about the effects of the planets' gravity on each other.

\textbf{Question 1.} Do you think that the gravity of Venus will have any effect on Earth's orbit? If so, how much? (You can answer ``None at all)


\vspace{1.5cm}
(2) \hrulefill\\


Kepler's first law says that planets orbit the Sun in elliptical orbits with the Sun at a focus. Does the orbit that you see follow Kepler's first law?


(10) \textbf{Question 2.} Now, try different starting velocities for the planet, and sketch the planet's orbit around its star. Notice that the starting point (1.02 AU) will always either be perihelion or aphelion. For each of the following, record:

\begin{itemize}

\item A picture of the planet's orbit, labeling its starting point, and mark where aphelion and perihelion are

\item How close, or how far, the planet gets to the Sun on the other side (the computer will tell you this; it's the other perihelion/aphelion point)

\item The eccentricity of the planet's orbit

\item Whether anything weird happened that wasn't at all what you expected

\end{itemize}


If your planet moves too slowly or too quickly, change the value of {\tt seconds\_per\_year} that controls how many seconds in the simulation corresponds to one year of real time.


\newpage
Try starting velocities of


\begin{enumerate}
	\item 2.5 AU/year
	\vspace{1.5in}
	\item 5 AU/year
	\vspace{1.5in}
	\item The value you found during your prelab that matches Earth's actual orbit
	\vspace{1.5in}
	\item 8 AU/year
	\vspace{1.5in}
	\item 10 AU/year
	
\end{enumerate}

\newpage

\textbf{Question 3.} What happened when you gave the Earth a large starting velocity of 10 AU/year? Is this what you expected? Do you think that the Earth would really move in this way if its velocity were suddenly increased to 10 AU/year?

\vspace{1.5cm}
(3) \hrulefill\\

\textbf {Question 4.} Remember that the focus of an ellipse is close to the center of the long axis if the ellipse is not very eccentric, but close to the edge if it is strongly eccentric. Look back at your drawings for the previous question. Is this true in the orbits you observed?

\vspace{1.5cm}
(2) \hrulefill\\

\textbf {Question 5.} Now, let's look at Kepler's second law, which compares {\it the speed of a planet} at {\it different places in its orbit}. By changing the starting distance from the Sun and the starting velocity, put your planet into an orbit with eccentricity at least 0.6.

Does your planet move more quickly near perihelion or aphelion? In evaluating this, look at both the animation and the graph to make sure that they agree. How does your graph show this?

\vspace{1.5cm}
(2) \hrulefill\\

\textbf {Question 6.} Does your planet spend more time near perihelion or aphelion?

\vspace{1.5cm}
(2) \hrulefill\\


\textbf {Question 7.} Does your planet appear to follow Kepler's second law? How do you know?

\vspace{1.5cm}
(3) \hrulefill\\


\newpage

Now, let's look at Kepler's third law, which compares {\it the time that different planets take to go around the Sun}.

Kepler's original formulation of the third law says that

$$
\frac{(\mbox{orbital period})^2}{(\mbox{length of long axis})^3} = \mbox{constant}
$$

This means that if you calculate this fraction for any planet orbiting the Sun, it should be the same.

\textbf {Question 8.} How can you determine the length of the long axis if you know the perihelion and aphelion distances? (A diagram on scratch paper may help you figure this out!)

\vspace{1.5cm}
(2) \hrulefill\\

\textbf{Question 9.} Simulate two different orbits with different starting distances and velocities. (Remember, if your planet flies off, your starting velocity is too high!)

\begin{center}
\Large
\begin{tabular}{| c | c | c | c | c |}
	\hline
	Aphelion & Perihelion & Long axis & Orbital period & $\frac{(\mbox{orbital period})^2}{(\mbox{long axis})^3}$ \\
	\hline
	& & & & \\[1.2cm]
	\hline
	& & & & \\[1.2cm]
	\hline
	\end{tabular}
\end{center}

(8) Do your two orbits appear to follow Kepler's third law? (You won't need to square and cube things on the exam, but you can do it here!)

\newpage

\section{Does the Mass Matter?}

The mass of the planet is set to the mass of Earth by default: 0.000003 times the mass of the Sun. Change the mass of the planet to the mass of Jupiter: 0.001 times the mass of the Sun. 

\textbf{Question 10.} Did the planet's orbit change? Describe how it changed, or explain why it didn't.

\vspace{1.5cm}

(2) \hrulefill\\

\textbf{Question 11.} Now, change the {\it star's} mass. Does this affect the orbit? Why or why not?

\vspace{1.5cm}

(2) \hrulefill\\

\vspace{3em}

Kepler's description of planet orbits -- where the Sun doesn't move and the planet orbits around it in an ellipse -- are based only on the motions of the planets in our solar system. They are true only if:

\begin{itemize}
	\item A much less massive object orbits a much more massive one (like Earth around the Sun)
	\item The gravity of the larger object being orbited is the only force that causes much of an effect on the smaller object. 
\end{itemize}

Earlier, you saw that changing the mass of the planet orbiting the Sun doesn't affect its orbit. But what happens if you make it {\it very} massive -- almost as massive as the Sun itself? 

\textbf{Question 12.} Increase the mass of the planet to at least ten percent of the star's mass. What happens {\it now?} Is this what you expect?

\vspace{1.5cm}

(2) \hrulefill\\


\textbf{Question 13.} Is this behavior something that matches Kepler's description of planetary orbits? If it's not, which of the assumptions above are no longer true?

\vspace{1.5cm}


(2) \hrulefill\\

\textbf{Question 14.} Suppose that you looked in space and saw a star wiggling back and forth. We can see stars with telescopes, but can't see planets since they don't make their own light. If you saw a star wiggling back and forth, what could you conclude about it?

\vspace{1.5cm}

(2) \textbf{Question 15.} We know that many stars are {\it binary stars} -- there are two stars that orbit each other closely. Since we know that gravity affects stars in the same way as it affects planets, we can use this program to simulate binary stars too. 

Set the mass of the ``planet'' to nearly the same as the mass of the ``star'': both of them will now represent stars. See if you can create an orbit where the two stars orbit each other. Show your orbit to your TA when you've got it.






\section{How's This All Work?}

This program may look complicated. But most of the code has to do with making the animation and table you see! The only code that actually performs the simulation -- where the actual {\it physics} is -- is between line 50 and 65.

All of the laws of nature that control the motion of orbits are the following:

\begin{itemize}
	\item Forces cause things to accelerate, where the acceleration is equal to the force divided by the mass of that thing {\bf (Newton's second law of motion)}
	\item Any two objects attract each other with a force equal to $F=\frac{Gm_1m_2}{r^2}$, where the force is directed toward the other object {\bf (The law of gravity)}
	\item Forces come in pairs: if the star's gravity exerts a force on the planet, the planet's gravity exerts a force back on the star that goes {\bf (Newton's third law of motion)} in the opposite direction and has the same size
	\item In a small interval of time, an object's position changes by an amount equal to its velocity multiplied by the amount of time. (You know this already: if you are traveling at 50 miles per hour, and you do this for two hours, you will go 100 miles...) {\bf (Definition of velocity)}
	\item In a small interval of time, an object's velocity changes by an amount equal to its acceleration multiplied by the amount of time. {\bf (Definition of acceleration)}
\end{itemize}
\newpage

The computer simulation in this program works as follows. First, we divide time up into very tiny intervals (the length is given on line 40, if you want to change it). Then, repeat the following:

\begin{enumerate}
	\item Determine the distance from the planet to the star
	\item Determine the {\it direction} from the planet to the star
	\item Using Newton's law of gravity and the distance and direction you just found, determine the force the two objects feel from each other
	\item Using Newton's second law of motion, determine the acceleration that this causes on each object
	\item Using the definition of velocity, change the positions of the star and the planet by the amount that they will move during the small interval
	\item Using the definition of acceleration, change the velocities of the star and the planet by the amount that they will change during the small interval
\end{enumerate}

This is all it does! 

\vspace{1em}

(5) \textbf{Question 16.} Look at the code from lines 50-65. Look at the code and try to identify what each group of lines does. Write those line numbers by the steps in the above list. 

(5) \textbf{Question 17.} As you saw earlier, this simulation has all of the features that Kepler observed in the orbits of the actual planets. However, Kepler's laws don't appear {\it anywhere} in this code; there are no mentions of ellipses, long and short axes, moving faster near perihelion, or anything like that!

You've already seen how this program works. Why do you think it correctly simulates the features of the planets' orbits, even though it doesn't have any of the substance of Kepler's laws inside it?








\end{document}