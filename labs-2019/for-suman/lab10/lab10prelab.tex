\documentclass[11pt]{article}
\usepackage{tocloft}
\usepackage{graphicx}
\usepackage{calc}
\usepackage{amssymb}
\usepackage{color}
\usepackage{array}
\usepackage[sc]{mathpazo}
\usepackage{url}
\usepackage[final]{pdfpages}
\usepackage{amsmath}

%\linespread{1.05}
\oddsidemargin=0pt
\evensidemargin=0pt
\textwidth=6.5in
\topmargin=0pt
\headheight=0pt
\headsep=0pt
\textheight=9in
% EXPERIMENTAL
%\parindent=0pt
%\parskip=3pt
\setlength{\parindent}{0cm}
\newcommand\secfont{\fontfamily{cmss}\selectfont}%\textwidth 5.5truein
\newcommand\pifheading[1]{{\secfont\textbf{#1}:}}
%\oddsidemargin -0.40truein
%\textheight 8.0truein
%\topmargin -0.25truein
\def\lo{
\mathrel{\raise.3ex\hbox{$<$}\mkern-14mu\lower0.6ex\hbox{$\sim$}}
}
\def\hi{
\mathrel{\raise.3ex\hbox{$>$}\mkern-14mu\lower0.6ex\hbox{$\sim$}}
}

\textwidth = 6.6 in
\textheight = 9.1 in
\oddsidemargin = -0.05 in
\evensidemargin = +0.05 in
\topmargin = -.1 in
\headheight = 0.0 in
\headsep = 0.0 in
\parskip = 0.06in
\newcommand\registered{{\ooalign{\hfil\raise .00ex\hbox{\scriptsize R}\hfil\crcr\mathhexbox20D}}}

%% Define a new 'leo' style for the package that will use a smaller font.
\makeatletter
\def\url@leostyle{%
  \@ifundefined{selectfont}{\def\UrlFont{\sf}}{\def\UrlFont{\small\ttfamily}}}
\makeatother
%% Now actually use the newly defined style.
\urlstyle{leostyle}

%\pagestyle{empty}
%\includeonly{previous,proposal_references}
%\includeonly{proposal_references}
%\includeonly{previous}

% TOC

\begin{document}
%%%%%%%%%%%%%%%%%%%%%%%%%%%%%%%%%%%%%%%%%%%%%%%%%%%%%%%%%%%%%%%%%%%%%
\begin{center}
\textbf{\Large
AST101: Our Corner of the Universe \\
\vspace*{0.1cm}
Lab 10: Radioactive Decay Prelab
}
\end{center}

\vspace*{0.5cm}

{\Large Name:}\vspace*{0.5cm}\\\hrule
{\Large Student number (SUID):}\vspace*{0.5cm}\\\hrule
{\Large Lab section:}\vspace*{0.5cm}\\\hrule
\vspace*{0.5cm}

%%%%%%%%%%%%%%%%%%%%%%%%%%%%%%%%%%%%%%%%%%%%%%%%%%%%%%%%%%%%%%%%%%%%%

The lab you will do simulates radioactive decay with a bunch of dice. You will roll the dice, pick out the ones that come
out 1, and repeat the process.

To get you used to thinking about the sorts of things you'll think about in the lab, answer the following questions
about doing the process with coins.

Specifically, imagine that you have a bunch of pennies. You flip them all, remove all the ones that come up heads, and 
then repeat the process.

{\bf Question 1:} If you flip eight pennies and remove all of the ones that come up heads, how many on average will you have
left?

\vspace{2in}

{\bf Question 2:} Suppose you did this twice in a row. How many pennies on average would you have left?
\vspace{2in}
\newpage
{\bf Question 3:} If you actually did this with eight pennies, would you be surprised if you were left with twice as many pennies as the average you calculated in Question 2? Half as many? 

\vspace{2in}

{\bf Question 4:} Suppose now you do this with a million pennies: flip them and take out the heads, then do that again. 
How many pennies will you, on average, have left?

\vspace{2in}

{\bf Question 5:} If you actually did this with a million pennies, would you be surprised if you were left with twice as many pennies as the average you calculated in Question 2? Half as many? 

\vspace{2in}

{\bf Question 6:} Suppose someone starts with 160 pennies and flips them some number of times, and then tells you that they
have 23 pennies left. How many times did they flip their collection of pennies?

\end{document}
